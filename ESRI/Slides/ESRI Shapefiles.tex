\documentclass{beamer}

\usepackage{amsmath}
\usepackage{graphicx}
\usepackage{framed}

\begin{document}

\begin{frame}
\Large
\begin{enumerate}
\item Create Maps With \texttt{R} Geospatial Classes and Graphics Tools (\textit{Making Maps})
\item Read and write ESRI Shape Files (\textit{ESRI})
\item Display T Spatial Objects with Google Maps and Google Earth (\textit{KML})
\item Read and Display Data from GPS Devices Using \texttt{R} (\textit{GPX})
\item Overlay Points on Satellite Image / Extract Pixel Values (\textit{Raster})
\end{enumerate}
\end{frame}
\begin{frame}
\huge
ESRI Shapefiles with \texttt{R}
\end{frame}
%---------------------------%
\begin{frame}
\frametitle{ESRI shapefiles}
\begin{itemize}
\item The ESRI Shapefile is a widely used file format for storing vector-based geopatial data (i.e., points, lines, and polygons).\item  This example demonstrates use of several different \texttt{R} packages that provide functions for reading and/or writing shapefiles.
\end{itemize}
\end{frame}
%---------------------------%
\begin{frame}
\frametitle{ESRI shapefiles}
\vspace{-1cm}
\textbf{Demonstration}\\

\begin{itemize}
\item The first two approaches  shown here use packages (specifically \textbf{rgdal} and \textbf{maptools}) that depend on the \textbf{sp} package, which defines a set of spatial classes that have become the de facto standard spatial data types in \texttt{R}. \item The third approach (PBS Mapping) creates \texttt{R} data objects that are less generally useful, but necessary when calling other analytical functions defined in the package.
\end{itemize} 
\end{frame}
%---------------------------%
\begin{frame}
\frametitle{Using rgdal}
\begin{itemize}
\item The rgdal package provides an interface to the \textit{\textbf{GDAL/OGR}} library, which powers the data import/export capabilities of 
many geospatially aware software applications. 
\item The package includes functions \texttt{readOGR} and \texttt{writeOGR} for reading and 
writing not only shapefiles, but numerous other vector-based file formats. 
\item In addition, the \texttt{ogrInfo} function is useful for retrieving details about the file without reading in the full dataset. These functions are all capable of automatically reading and writing projection information if available. 
\end{itemize} 
\end{frame}
%---------------------------%
\begin{frame}
\frametitle{Using rgdal}
\vspace{-1cm}
\begin{itemize}
\item Provided you are able to install the separate \textbf{GDAL/OGR} library - which may be tricky on some systems - it is worth learning how to use this package if you frequently work with shapefiles and/or other spatial data formats, including not just vector formats but raster formats as well.
\end{itemize}
\end{frame}
%---------------------------%
\begin{frame}
\frametitle{Using maptools:}
\begin{itemize}
\item The \textbf{maptools} package includes a number of useful functions for reading, writing, converting, and otherwise handling 
spatial objects in \textit{R}. 
\item The general functions for reading and writing shapefiles are \texttt{readShapeSpatial} and \texttt{writeSpatialShape}, respectively. 
\item 
In both cases, the function automatically determines whether the shapefile (or \texttt{R} object) contains points, 
lines, or polygons, and will then read in (or write out) the data using a more specialized function of the particular type. 
\end{itemize}
\end{frame}
%---------------------------%
\begin{frame}
\frametitle{Using maptools}
\begin{itemize}
\item These specialized functions, such as re\texttt{adShapeLines} for reading lines, can also be called directly. 
\item 
One advantage of doing so is that it will complain if you inadvertently use it on the wrong data type, helping you to catch errors sooner. \item Unlike their \textbf{rgdal} counterparts, the ma\textbf{ptools} functions neither read nor write projection information, leaving it up to you to manage these details manually.
\end{itemize}
\end{frame}

\begin{frame}
\frametitle{About PBSMapping (CRAN Description)}
\textbf{PBSmapping:\\ Mapping Fisheries Data and Spatial Analysis Tools:}\\

This software has evolved from fisheries research conducted at the \emph{\textbf{Pacific Biological Station}} (PBS) in Nanaimo, British Columbia, Canada. It extends the \texttt{R} language to include two-dimensional plotting features similar to those commonly available in a Geographic Information System (GIS). 

Embedded C code speeds algorithms from computational geometry, such as finding polygons that contain specified point events or converting between longitude-latitude and Universal Transverse Mercator (UTM) coordinates.
\end{frame}
%---------------------------%
\begin{frame}
\frametitle{Using PBSmapping}
\begin{itemize}
\item The PBSmapping package can also read (but not write) shapefiles. 
\item However, note that \textbf{PBSmapping} uses its own custom-defined spatial data types that are optimized to work with various 
specialized package functions. \item  This makes it harder to take advantage of functions defined in the numerous packages 
that are built on \textbf{sp}, although the \textbf{\textit{maptools}} package does provide functions that convert between the different formats.
\end{itemize}
\end{frame}
%---------------------------%



\end{document}
