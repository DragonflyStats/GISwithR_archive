

Scale bar and North arrow on a ggplot2 map using R
Posté le 10 novembre 2013 par Ewen
After some research on the Internet, I gave up trying to find an R function to add a scale bar and a North arrow on a map, using ggplot().
So, I would like to present you what I have come up with.

The idea is really basic : we create two polygons for the scale bar, we add some text above, and we draw an arrow. That’s it. The only tricky thing here, is the way to obtain the coordinates of each element. An easy solution is to use the gcDestination function, from the maptools package.

From the user’s point of vue, there is only one function to use (scaleBar), and a few arguments to pass. However, this function has some dependencies.

First, let us load some packages:

?

library(maps)
library(maptools)
library(ggplot2)
library(grid)
Then, we need a function to get the scale bar coordinates:

?

#
# Result #
#--------#
# Return a list whose elements are :
#   - rectangle : a data.frame containing the coordinates to draw the first rectangle ;
#   - rectangle2 : a data.frame containing the coordinates to draw the second rectangle ;
#   - legend : a data.frame containing the coordinates of the legend texts, and the texts as well.
#
# Arguments : #
#-------------#
# lon, lat : longitude and latitude of the bottom left point of the first rectangle to draw ;
# distanceLon : length of each rectangle ;
# distanceLat : width of each rectangle ;
# distanceLegend : distance between rectangles and legend texts ;
# dist.units : units of distance "km" (kilometers) (default), "nm" (nautical miles), "mi" (statute miles).
createScaleBar <- function(lon,lat,distanceLon,distanceLat,distanceLegend, dist.units = "km"){
    # First rectangle
    bottomRight <- gcDestination(lon = lon, lat = lat, bearing = 90, dist = distanceLon, dist.units = dist.units, model = "WGS84")
     
    topLeft <- gcDestination(lon = lon, lat = lat, bearing = 0, dist = distanceLat, dist.units = dist.units, model = "WGS84")
    rectangle <- cbind(lon=c(lon, lon, bottomRight[1,"long"], bottomRight[1,"long"], lon),
    lat = c(lat, topLeft[1,"lat"], topLeft[1,"lat"],lat, lat))
    rectangle <- data.frame(rectangle, stringsAsFactors = FALSE)
     
    # Second rectangle t right of the first rectangle
    bottomRight2 <- gcDestination(lon = lon, lat = lat, bearing = 90, dist = distanceLon*2, dist.units = dist.units, model = "WGS84")
    rectangle2 <- cbind(lon = c(bottomRight[1,"long"], bottomRight[1,"long"], bottomRight2[1,"long"], bottomRight2[1,"long"], bottomRight[1,"long"]),
    lat=c(lat, topLeft[1,"lat"], topLeft[1,"lat"], lat, lat))
    rectangle2 <- data.frame(rectangle2, stringsAsFactors = FALSE)
     
    # Now let's deal with the text
    onTop <- gcDestination(lon = lon, lat = lat, bearing = 0, dist = distanceLegend, dist.units = dist.units, model = "WGS84")
    onTop2 <- onTop3 <- onTop
    onTop2[1,"long"] <- bottomRight[1,"long"]
    onTop3[1,"long"] <- bottomRight2[1,"long"]
     
    legend <- rbind(onTop, onTop2, onTop3)
    legend <- data.frame(cbind(legend, text = c(0, distanceLon, distanceLon*2)), stringsAsFactors = FALSE, row.names = NULL)
    return(list(rectangle = rectangle, rectangle2 = rectangle2, legend = legend))
}
We also need a function to obtain the coordinates of the North arrow:


#
# Result #
#--------#
# Returns a list containing :
#   - res : coordinates to draw an arrow ;
#   - coordinates of the middle of the arrow (where the "N" will be plotted).
#
# Arguments : #
#-------------#
# scaleBar : result of createScaleBar() ;
# length : desired length of the arrow ;
# distance : distance between legend rectangles and the bottom of the arrow ;
# dist.units : units of distance "km" (kilometers) (default), "nm" (nautical miles), "mi" (statute miles).
createOrientationArrow <- function(scaleBar, length, distance = 1, dist.units = "km"){
    lon <- scaleBar$rectangle2[1,1]
    lat <- scaleBar$rectangle2[1,2]
     
    # Bottom point of the arrow
    begPoint <- gcDestination(lon = lon, lat = lat, bearing = 0, dist = distance, dist.units = dist.units, model = "WGS84")
    lon <- begPoint[1,"long"]
    lat <- begPoint[1,"lat"]
     
    # Let us create the endpoint
    onTop <- gcDestination(lon = lon, lat = lat, bearing = 0, dist = length, dist.units = dist.units, model = "WGS84")
     
    leftArrow <- gcDestination(lon = onTop[1,"long"], lat = onTop[1,"lat"], bearing = 225, dist = length/5, dist.units = dist.units, model = "WGS84")
     
    rightArrow <- gcDestination(lon = onTop[1,"long"], lat = onTop[1,"lat"], bearing = 135, dist = length/5, dist.units = dist.units, model = "WGS84")
     
    res <- rbind(
            cbind(x = lon, y = lat, xend = onTop[1,"long"], yend = onTop[1,"lat"]),
            cbind(x = leftArrow[1,"long"], y = leftArrow[1,"lat"], xend = onTop[1,"long"], yend = onTop[1,"lat"]),
            cbind(x = rightArrow[1,"long"], y = rightArrow[1,"lat"], xend = onTop[1,"long"], yend = onTop[1,"lat"]))
     
    res <- as.data.frame(res, stringsAsFactors = FALSE)
     
    # Coordinates from which "N" will be plotted
    coordsN <- cbind(x = lon, y = (lat + onTop[1,"lat"])/2)
     
    return(list(res = res, coordsN = coordsN))
}
The last function enables the user to draw the elements:


#
# Result #
#--------#
# This function enables to draw a scale bar on a ggplot object, and optionally an orientation arrow #
# Arguments : #
#-------------#
# lon, lat : longitude and latitude of the bottom left point of the first rectangle to draw ;
# distanceLon : length of each rectangle ;
# distanceLat : width of each rectangle ;
# distanceLegend : distance between rectangles and legend texts ;
# dist.units : units of distance "km" (kilometers) (by default), "nm" (nautical miles), "mi" (statute miles) ;
# rec.fill, rec2.fill : filling colour of the rectangles (default to white, and black, resp.);
# rec.colour, rec2.colour : colour of the rectangles (default to black for both);
# legend.colour : legend colour (default to black);
# legend.size : legend size (default to 3);
# orientation : (boolean) if TRUE (default), adds an orientation arrow to the plot ;
# arrow.length : length of the arrow (default to 500 km) ;
# arrow.distance : distance between the scale bar and the bottom of the arrow (default to 300 km) ;
# arrow.North.size : size of the "N" letter (default to 6).
scaleBar <- function(lon, lat, distanceLon, distanceLat, distanceLegend, dist.unit = "km", rec.fill = "white", rec.colour = "black", rec2.fill = "black", rec2.colour = "black", legend.colour = "black", legend.size = 3, orientation = TRUE, arrow.length = 500, arrow.distance = 300, arrow.North.size = 6){
    laScaleBar <- createScaleBar(lon = lon, lat = lat, distanceLon = distanceLon, distanceLat = distanceLat, distanceLegend = distanceLegend, dist.unit = dist.unit)
    # First rectangle
    rectangle1 <- geom_polygon(data = laScaleBar$rectangle, aes(x = lon, y = lat), fill = rec.fill, colour = rec.colour)
     
    # Second rectangle
    rectangle2 <- geom_polygon(data = laScaleBar$rectangle2, aes(x = lon, y = lat), fill = rec2.fill, colour = rec2.colour)
     
    # Legend
    scaleBarLegend <- annotate("text", label = paste(laScaleBar$legend[,"text"], dist.unit, sep=""), x = laScaleBar$legend[,"long"], y = laScaleBar$legend[,"lat"], size = legend.size, colour = legend.colour)
     
    res <- list(rectangle1, rectangle2, scaleBarLegend)
     
    if(orientation){# Add an arrow pointing North
        coordsArrow <- createOrientationArrow(scaleBar = laScaleBar, length = arrow.length, distance = arrow.distance, dist.unit = dist.unit)
        arrow <- list(geom_segment(data = coordsArrow$res, aes(x = x, y = y, xend = xend, yend = yend)), annotate("text", label = "N", x = coordsArrow$coordsN[1,"x"], y = coordsArrow$coordsN[1,"y"], size = arrow.North.size, colour = "black"))
        res <- c(res, arrow)
    }
    return(res)
}
Now, let’s play with this!

Let us draw the US map:

?
1
2
3
# United States map
usa.map <- map_data("state")
P <- ggplot() + geom_polygon(data = usa.map, aes(x = long, y = lat, group = group)) + coord_map()
If we want to add the scale bar only:

?
1
P + scaleBar(lon = -130, lat = 26, distanceLon = 500, distanceLat = 100, distanceLegend = 200, dist.unit = "km", orientation = FALSE)

US map using ggplot2 and scaleBar, without North arrow

Or if we want the scale bar and the North arrow:
?
1
P <- P + scaleBar(lon = -130, lat = 26, distanceLon = 500, distanceLat = 100, distanceLegend = 200, dist.unit = "km")
To make it look more like a map:

?

P + theme(panel.grid.minor = element_line(colour = NA), panel.grid.minor = element_line(colour = NA),
    panel.background = element_rect(fill = NA, colour = NA), axis.text.x = element_blank(),
    axis.text.y = element_blank(), axis.ticks.x = element_blank(),
    axis.ticks.y = element_blank(), axis.title = element_blank(),
    rect = element_blank(),
    plot.margin = unit(0 * c(-1.5, -1.5, -1.5, -1.5), "lines"))

US map using ggplot2 and scaleBar
Another example, with France:

?

france.map <- map_data("france")
P.fr <- ggplot() + geom_polygon(data = france.map, aes(x = long, y = lat, group = group)) + coord_map()
 
# Let us add the scale bar
P.fr <- P.fr + scaleBar(lon = -5, lat = 42.5, distanceLon = 100, distanceLat = 20, distanceLegend = 40, dist.unit = "km", arrow.length = 100, arrow.distance = 60, arrow.North.size = 6)
 
# Modifying the theme a bit
P.fr + theme(panel.grid.minor = element_line(colour = NA), panel.grid.minor = element_line(colour = NA),
    panel.background = element_rect(fill = NA, colour = NA), axis.text.x = element_blank(),
    axis.text.y = element_blank(), axis.ticks.x = element_blank(),
    axis.ticks.y = element_blank(), axis.title = element_blank(),
    rect = element_blank(),
    plot.margin = unit(0 * c(-1.5, -1.5, -1.5, -1.5), "lines"))

France map using ggplot2 and scaleBar
Posté dans Cartes, Informatique, Non classé	 | Tags : ggplot2, legend, Map, north arrow, R, scale bar	 |
