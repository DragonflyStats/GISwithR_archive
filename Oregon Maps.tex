\documentclass[12pt]{report}

\usepackage{amsmath}
\usepackage{amssymb}
\usepackage{graphics}
\usepackage{framed}
\begin{document}
\section*{Data sets}
\begin{description}
\item[orstationc.csv] -- Oregon climate station data (includes lat and lon)
\item[orcountyp.csv] -- Oregon county census data (includes lats and lons, but not outlines)
\item[cities2.csv] -- Large cities data set, 
    with country names added
\end{description}

\begin{framed}
\begin{verbatim}
# Load the libraries:

library(gpclib)
library(maptools)     # loads sp library too
library(RColorBrewer) # creates nice color schemes
library(classInt)     # finds class intervals for continuous variables
\end{verbatim}
\end{framed}
To enable the polygon-clipping library, type:
\texttt{gpclibPermit()}.


Read the shapefiles using the maptools function \texttt{read.shape()}
\begin{framed}
\begin{verbatim}

# outlines of Oregon counties (lines)
# browse to orotl.shp
orotl.shp <- readShapeLines(file.choose(),
     proj4string=CRS("+proj=longlat"))

# Oregon climate station data (points)
# browse to orstations.shp
orstations.shp <- readShapePoints(file.choose(),
     proj4string=CRS("+proj=longlat"))
\end{verbatim}
\end{framed}
     
\begin{framed}
\begin{verbatim}
# Oregon county census data (polygons)
# browse to orcounty.shp
orcounty.shp <- readShapePoly(file.choose(),
     proj4string=CRS("+proj=longlat"))
Read ordinary rectangular data sets:
orstationc <- read.csv("orstationc.csv")
orcountyp <- read.csv("orcountyp.csv")
cities <- read.csv("cities.csv")
\end{verbatim}
\end{framed}
Examine the structure and contents of orcounty.shp shapefile:

\begin{framed}
\begin{verbatim}
summary(orcounty.shp)
attributes(orcounty.shp)
attributes(orcounty.shp@data)
attr(orcounty.shp,"polygons")

\end{verbatim}
\end{framed}
\section{Some simple maps}
\texttt{R} has the capability of plotting some simple maps using the \textit{\textbf{maptools}} package, which can read and plot ESRI shapefiles.  

Here are a couple of examples:
Oregon county census data -- attribute data in the \textbf{\textit{orcounty.shp}} shape file
%------------------------------------------------------------------%
\begin{framed}
\begin{verbatim}
# equal-frequency class intervals
plotvar <- orcounty.shp@data$POP1990
nclr <- 8
plotclr <- brewer.pal(nclr,"BuPu")
class <- classIntervals(plotvar, nclr, style="quantile")
colcode <- findColours(class, plotclr)

plot(orcounty.shp, xlim=c(-124.5, -115), ylim=c(42,47))
plot(orcounty.shp, col=colcode, add=T)
title(main="Population 1990",
    sub="Quantile (Equal-Frequency) Class Intervals")
legend(-117, 44, legend=names(attr(colcode, "table")),
    fill=attr(colcode, "palette"), cex=0.6, bty="n")

\end{verbatim}
\end{framed}
%------------------------------------------------------------------%
Oregon climate station data -- data in the orstationc.csv file, basemap in orotl.shp
\begin{framed}
\begin{verbatim}


# symbol plot -- equal-interval class intervals
plotvar <- orstationc$tann
nclr <- 8
plotclr <- brewer.pal(nclr,"PuOr")
plotclr <- plotclr[nclr:1] # reorder colors
class <- classIntervals(plotvar, nclr, style="equal")
colcode <- findColours(class, plotclr)
\end{verbatim}
\end{framed}
%------------------------------------------------------------------%
\begin{framed}
\begin{verbatim}
plot(orotl.shp, xlim=c(-124.5, -115), ylim=c(42,47))
points(orstationc$lon, orstationc$lat, 
    pch=16, col=colcode, cex=2)
points(orstationc$lon, orstationc$lat, cex=2)
title("Oregon Climate Station Data -- Annual Temperature",
    sub="Equal-Interval Class Intervals")
legend(-117, 44, legend=names(attr(colcode, "table")),
    fill=attr(colcode, "palette"), cex=0.6, bty="n")

\end{verbatim}
\end{framed}
Oregon climate station data -- locations and data in shape file
%------------------------------------------------------------------%
\begin{framed}
\begin{verbatim}
# symbol plot -- equal-interval class intervals
plotvar <- orstations.shp@data$pann
nclr <- 5
plotclr <- brewer.pal(nclr,"BuPu")
class <- classIntervals(plotvar, nclr, style="fixed",
fixedBreaks=c(0,200,500,1000,2000,5000))
colcode <- findColours(class, plotclr)
orstations.pts <- orstations.shp@coords # get point data
\end{verbatim}
\end{framed}
%------------------------------------------------------------------%
\begin{framed}
\begin{verbatim}

plot(orotl.shp, xlim=c(-124.5, -115), ylim=c(42,47))
points(orstations.pts, pch=16, col=colcode, cex=2)
points(orstations.pts, cex=2)
title("Oregon Climate Station Data -- Annual Precipitation",
    sub="Fixed-Interval Class Intervals")
legend(-117, 44, legend=names(attr(colcode, "table")),
fill=attr(colcode, "palette"), cex=0.6, bty="n")

\end{verbatim}
\end{framed}
%------------------------------------------------------------------%

\end{document}